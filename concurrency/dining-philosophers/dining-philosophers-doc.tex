% Created 2016-11-06 dom 20:36
\documentclass{article}
\usepackage[utf8]{inputenc}
\usepackage[T1]{fontenc}
\usepackage{fixltx2e}
\usepackage{graphicx}
\usepackage{longtable}
\usepackage{float}
\usepackage{wrapfig}
\usepackage{rotating}
\usepackage[normalem]{ulem}
\usepackage{amsmath}
\usepackage{textcomp}
\usepackage{marvosym}
\usepackage{wasysym}
\usepackage{amssymb}
\usepackage{hyperref}
\tolerance=1000
\include{common-def}
\include{common-packages}
\include{common-pdf-setup}
\author{Luca Manini}
\date{2014}
\title{Filosofi a cena: esempio di programmazione concorrente}
\hypersetup{
  pdfkeywords={},
  pdfsubject={},
  pdfcreator={Emacs 24.3.1 (Org mode 8.2.4)}}
\begin{document}

\maketitle
\section{Introduzione}
\label{sec-1}

Classico problema di concorrenza, legato alla condivisione di
risorse e all'accesso condiviso a qualche informazione
"aggiornabile".  Il problema è il seguente: ci sono cinque filosofi
a cena, che passano la serata alternando momenti in cui pensano
(\emph{think}) a momenti in cui mangiano (\emph{eat}).  Ogni filosofo ha il
suo piatto (spaghetti), ma ci sono solo cinque forchette (ai lati
dei piatti) e ogni filosofo ha bisogno delle due forchette al lato
dei suo piatto per mangiare! Morale: solo due filosofi, non
"contigui" possono mangiare contemporaneamente.

La prima soluzione che viene in mente è: quando un filosofo vuole
mangiare guarda se le due posate sono disponibili e se lo sono le
prende e mangia per un po' (togliendo quindi ai suoi vicini la
possibilità di mangiare) e poi posa le posate.  

Il problema è che, essendo i cinque filosofi processi "paralleli",
può succedere che tutti vedono le posate disponibili, tutti prendono
quella alla loro sinistra e poi non possono però prendere quella a
destra: sono tutti bloccati.  Il nocciolo del problema è che le due
azioni di controllare la disponibilità delle due forchette e di
prenderle devono costituire un'unica azione "atomica" (proprio nel
senso di indivisibile) che o ha successo globalmente o fallisce
globalmente, senza mai permettere che la prima delle due abbia
successo e che la seconda fallisca.  La ragione fondamentale per cui
la seconda può fallire e che tra l'esecuzione della prima e quella
della seconda si "inserisca" un'azione di un altro filosofo; la
soluzione sta quindi nel "sincronizzare" in modo intelligente le
varie operazioni, limitando in un certo senso il parallelismo.

\section{La soluzione classica}
\label{sec-2}


La soluzione classica è quella proposta da Dijstra nel 1956 e
riportata in tutti i libri di testo, tra i quali il libro \emph{Modern
Operating Systems} di Andrew S. Tanenbaum (AST nel seguito) da cui
ho preso la soluzione che illustro di seguito (un po' modificata).

Un esempio di "non soluzione", che non credo richieda spiegazioni, è
il seguente:
\begin{verbatim}
#define N 5                 /* number of philosophers */
void philosopher (int i) {

  while (1) {

    think();
    take_fork(i);
    take_fork((i+1) % N);
    eat();
    put_fork(i);
    put_fork((i+1) % N);
  }
}
\end{verbatim}

Un primo miglioramento potrebbe essere quello di proteggere tutto il
corpo del \texttt{for}, esclusa la chiamata a \texttt{think} da un semaforo, ma
così al massimo un solo filosofo potrebbe mangiare in ogni momento,
mentre è ovvio che con cinque forchette si potrebbe arrivare a due!

La soluzione è avere:

\begin{enumerate}
\item un vettore che mantiene lo stato (\emph{thinking', /hungry} e
\emph{eating}) di ciascun filosofo;
\item un singolo \emph{mutex} per proteggere le \emph{critical region} in cui si
accede al vettore stesso;
\item un vettore di semafori su cui i filosofi possono "bloccarsi" in
attesa della disponibilità di forchette.
\end{enumerate}

\section{L'implementazione}
\label{sec-3}

L'implementazione di AST è composta da cinque parti fondamentali:

\begin{enumerate}
\item le strutture dati;
\item la funzione \texttt{philosopher};
\item la funzione \texttt{take\_forks};
\item la funzione \texttt{put\_forks};
\item la funzione \texttt{test}.
\end{enumerate}

\subsection{Le strutture dati}
\label{sec-3-1}

Ciò che serve sono i \texttt{define} di alcune costanti e le due macro
\texttt{LEFT} e \texttt{RIGHT} per ottenere facilemente gli indici dei due
filosofi "vicini".  Tanenbaum usa gli interi come semafori, mentre
io uso il tipo \texttt{sem\_t} (predefinito in \texttt{thread.h}); per questo può
inizializzare il \emph{mutex} direttamente a 1, mentre io lo faccio nel
\texttt{main} con la funzione \texttt{up} (\emph{alias} di \texttt{sem\_post}).

\begin{verbatim}
#define N 5
#define LEFT  (i-1) % N
#define RIGHT (i+1) % N
#define THINKING 0
#define HUNGRY   1
#define EATING   2
typedef int semaphore;
semaphore mutex = 1;
semaphore s[N];       
int state[N];
\end{verbatim}

\subsection{La funzione \texttt{philosopher}}
\label{sec-3-2}

La funzione \texttt{philosopher} è banale, perché tutte le difficoltà sono
nascoste in \texttt{take\_forks} e \texttt{put\_forks}.  Nella mia implementazione,
\texttt{think} e \texttt{eat} accettano un argomento (l'indice \texttt{i}) per
permettere il \emph{logging}.

\begin{verbatim}
void philosopher(int i) {

  while (1) {           
    think();
    take_forks(i);
    eat();
    put_forks(i);
  }
}
\end{verbatim}

\subsection{La funzione \texttt{take\_forks}}
\label{sec-3-3}

Qui cominciano le difficoltà! Vediamo di capire cosa succede:

\begin{enumerate}
\item le due chiamate su \texttt{mutex} servono a proteggere la sezione
critica, ossia l'accesso diretto o indiretto (via \texttt{test}) al
vettore \texttt{state};
\item il filosofo prima di cambia il suo stato in \texttt{HUNGRY} e poi
chiama la funzione \texttt{test} che dopo controlla se è il caso e se è
possibile prendere le forchette (ovvero se i due filosofi vicini
non stanno mangiando) e se tutto va bene passa allo stato
\texttt{EATING} e fa un \texttt{up} su \texttt{s[i]};
\item poi il filosofo esegue una \texttt{down} su \texttt{s[i]} che lo blocca se
nella \texttt{test} non è stato eseguita la \texttt{up} corrispondente (verrà
poi eventualmente sboccato dai vicini, vedremo come).
\end{enumerate}

\begin{verbatim}
void take_forks(int i) {

  down(&mutex);
  state[i] = HUNGRY;
  test(i);
  up(&mutex);
  down(&s[i]);
}
\end{verbatim}

\subsection{La funzione \texttt{put\_forks}}
\label{sec-3-4}

La funzione \texttt{put\_forks} è relativamente semplice, le due chiamate a
\texttt{test} servono a "sbloccare" (se ce ne fosse bisogno) i due vicini.

\begin{verbatim}
void put_forks(int i) {

  down(&mutex);
  state[i] = THINKING;
  test(LEFT);
  test(RIGHT);
  up(&mutex);
}
\end{verbatim}

\subsection{La funzione \texttt{test}}
\label{sec-3-5}

La funzione \texttt{test} è forse quella più "magica" e più difficile da
capire (anche perché il nome non è forse scelto in modo ottimale).
In pratica \emph{offre} ad un filosofo la possibilità di mangiare a
patto che sia nello stato \texttt{HUNGRY} e che i suoi vicini non stiamo
mangiando.  È importante notare che la funzione è chiamata da un
filosofo in due "modi" e in due "occasioni" differenti:

\begin{enumerate}
\item su sé stesso nella \texttt{take\_forks},
\item sui vicini nella \texttt{put\_forks}.
\end{enumerate}

\begin{verbatim}
void test(int i) {

  if (state[i]     == HUNGRY &&
      state[LEFT]  != EATING &&
      state[RIGHT] != EATING) {

    state[i] = EATING;
    up(&s[i]);    
  }
}
\end{verbatim}


\section{Una ricostruzione (in Python)}
\label{sec-4}


Ammesso che la soluzione di Dijstra non mi sarebbe mai venuta in
mente, ma che però ho capito come funziona, adesso voglio vedere di
trovare un "racconto" di una sua "ricostruzione" e già che ci sono
usando Python!

Comincio con un parametro che indica il numero dei filosofi, una
lista che mantenga lo stato di ciascuno di loro e un semaforo che
protegga l'accesso alla lista (inizialmente disponibile).
\begin{verbatim}
import threading
import random
import time
import sys

PHIL_COUNT = 5
THINKING, HUNGRY, EATING = range(3)
STATE = [THINKING] * PHIL_COUNT
SEM_STATE = threading.Semaphore(1)
\end{verbatim}

In molti punti dell'implementazione devo fare riferimento ad un dato
filosofo e ai due che siedono alla sua sinistra e alla sua destra.
Siccome se faccio un disegno della tavola numero i filosofi in senso
orario, mi viene naturale usare l'indice successivo per il filosofo
di sinistra e il precedente per quello di destra.  Definisco delle
funzioni che implementano questa scelta:
\begin{verbatim}
def LEFT(i):
    return (i + 1) % PHIL_COUNT

def RIGHT(i):
    return (i - 1) % PHIL_COUNT
\end{verbatim}

Definisco alcune funzioni per la gestione dei semafori.  Preferisco
i nomi \texttt{wait} e \texttt{signal} quando uso i semafori per sincronizzare
\emph{thread} distinti, mentre trovo più espressivi \texttt{acquire} e \texttt{release}
quando li uso come "lock" per proteggere delle \emph{critical section}.
\begin{verbatim}
def wait(sem):
    sem.acquire()

def signal(sem):
    sem.release()

acquire = wait
release = signal
\end{verbatim}

Poi posso scrivere le funzione \texttt{philosopher}, \texttt{think} e \texttt{eat} che
sono abbastanza ovvie e le funzioni accessorie \texttt{get\_think\_time} e
\texttt{get\_eat\_time} almeno in una loro versione banale.
\begin{verbatim}
def get_think_time(i):
  return random.random()

def get_eat_time(i):
  return random.random()

def show(s):
    sys.stderr.write(s + "\n")

def think(i):
  show("THINK %d" % i)
  t = get_think_time(i)
  time.sleep(t)

def eat(i):
  show("EAT %d" % i)
  t = get_eat_time(i)
  time.sleep(t)

RUN = True

def philosopher(i):

  while RUN:
      think(i)
      take_forks(i)
      eat(i)
      put_forks(i)
\end{verbatim}

Adesso comincia la parte difficile: ricostruire \texttt{take\_forks} e
\texttt{put\_forks}.

Forse è più facile cominciare da \texttt{put\_forks}, per la quale la
successione delle operazioni mi pare più intuitiva.  Il filosofo si
può mettere subito nello stato \texttt{THINKING} per poi offrire ai suoi
due vicini la possibilità di mangiare.
\begin{verbatim}
def put_forks(i):
    acquire(SEM_STATE)
    STATE[i] = THINKING
    test_chance(LEFT(i))
    test_chance(RIGHT(i))
    release(SEM_STATE)
\end{verbatim}

Ora il problema è cosa mettere in \texttt{test\_chance} (che è la \texttt{test}
della soluzione originale).  Si può intuire che dovrà includere una
\texttt{signal} su un semaforo su cui i vicini potrebbero essere in \texttt{wait}.
Mi serve quindi una lista \texttt{TURN} di semafori.  
\begin{verbatim}
TURN = [threading.Semaphore(0) for _ in range(PHIL_COUNT)]
\end{verbatim}

D'altra parte un filosofo può sfruttare l'offerta solo se è nello
stato \texttt{HUNGRY} e se i \textbf{suoi} vicini non stanno mangiando!  Una prima
approssimazione potrebbe quindi essere la seguente:
\begin{verbatim}
def test_chance(i):
    if (STATE[i] == HUNGRY and
        STATE[LEFT(i)] != EATING and
        STATE[RIGHT(i)] != EATING):

        signal(TURN[i])
\end{verbatim}

Provo adesso a scrivere \texttt{take\_forks}.  La prima cosa da fare è
"prenotarsi" per una possibile \emph{chance} passando nello stato
\texttt{HUNGRY}, poi provare la fortuna con \texttt{test\_chance} e mettersi in
\texttt{wait}.  Il trucco qui è che mi metto in \texttt{wait} su un semaforo che,
forse, è stato appena incrementato (nella \texttt{test\_chance}).
\begin{verbatim}
def take_forks(i):
    acquire(SEM_STATE)
    STATE[i] = HUNGRY
    test_chance(i)
    release(SEM_STATE)
    wait(TURN[i])
\end{verbatim}

Resta il problema di dove fare il passaggio a \texttt{EATING}!  Ovviamente
devo farlo all'interno di una \emph{critical section} protetta da
\texttt{SEM\_STATE}, ma d'altra parte la \texttt{wait} ne deve rimanere fuori
(altrimenti rischio di bloccare tutto).  Potrei quindi pensare di
passare a \texttt{EATING} all'interno di \texttt{test\_chance}!
\begin{verbatim}
def test_chance(i):
    if (STATE[i] == HUNGRY and
        STATE[LEFT(i)] != EATING and
        STATE[RIGHT(i)] != EATING):

        STATE[i] = EATING
        signal(TURN[i])
        ii = [str(i) for i in range(PHIL_COUNT) if STATE[i] == EATING]
        s = ", ".join(ii)
        show("Eaters: %s" % s)
        if len(ii) > 2:         
            raise Exception("Cazzarola")
\end{verbatim}

Adesso serve un minimo di \texttt{main} per provare il tutto!

\begin{verbatim}
def main():

    tt = [threading.Thread(target=philosopher,
                           args=(i,))
                           for i in range(PHIL_COUNT)]
    for t in tt:
        t.start()

    for t in tt:
        t.join()

if __name__ == '__main__':

    main()
\end{verbatim}
% Emacs 24.3.1 (Org mode 8.2.4)
\end{document}